\documentclass[iop,apj]{emulateapj}
\usepackage{amsmath,amssymb,amstext}

\usepackage[breaklinks,colorlinks,citecolor=blue,linkcolor=magenta]{hyperref} 
\renewcommand*{\sectionautorefname}{Section}
\usepackage[all]{hypcap} %Links go to figures; breaks on deluxetables (use \capstartfalse \capstarttrue to fix it)

\usepackage{aas_macros}
\usepackage{natbib}
\bibliographystyle{apj}

\shorttitle{Short Title}
\shortauthors{Author et al.}

\begin{document}

\title{Detecting Active Asteroids/Comets from OSSOS survey images}
\author{Authors}
\affil{Affiliations}

\begin{abstract}
Abstract.
\end{abstract}

\keywords{keywords}
\maketitle

%Introduce what comets are, their physical processes
%	- how they are detected
%	- survey process, OSSOS, CFHT

\section{Introduction}
% so far, im not making a point of using good wording
% definietly have to go back and add in citations
The active asteroids (also known as main belt comets) are small bodies in the main asteroid belt which have prolonged or periodic dust emission producing comet-like comae and tails. Unlike comets, which originate in the Kuiper Belt and Oort cloud and have been scattered inwards by gravitational affects with the large planets, the active asteroids have been stable orbits confined to the main belt, and likely formed in the same location as they reside presently \cite{pop of comets 2006}. For objects which formed in the the outer region of the main belt, beyond the snow line, the crystallized water ice which was present at the time of formation and not exposed to primordial heating, may still remain in reservoirs beneath the surface \citep{limits on size, prialnik 2009}. If the ice were to be exposed due to an event such as an impact, rotational instability driven by YORP torques, or heating during perihelion passage could trigger sublimation which ejects dust particles from the surface producing a comae. Radiation pressure affects would then (preferentially) push the smaller particles away from the asteroid forming a tail.

Sublimation driven active asteroids differ from comets in that the comets, having larger ice reservoirs, would have stronger sublimation events which could eject larger debris. The cometary tail would then be longer lived as the larger debris would be less effected by the radiation pressure, and thus dissipate slowly. It is also possible, however, that prolonged activity is a result of ongoing ejection of small, fast dissipating particles. \cite{cite this}




%What samples we had
%	- working with data taken from 2013-present from OSSOS survey
%		- objects with known orbital path, therefore can predict locations in images
%		- limited by objects that are big/bright enough to detect
%	- cuts: sufficient number of images
%	- identify objects: predicted location, magnitude, ellipse shape, (psf?)
%	
%	- identify AA/Comet by transient trail/comae


%How well the serendipitous asteroids sample the whole population
%	- what fraction of total asteroids were imaged?
%	- what fraction of imaged asteroids were identified as AA/Comets?
%		- what is the distinguishing factor?



Detection of the coma or tails is highly dependent on the magnitude constraints of the survey for small dark objects. As most asteroids fall near the limiting magnitudes of the survey in which they are discovered \cite{active asteroids}, objects which are larger, closer, or have higher albedo are preferentially detected, and dust emission would be more easily apparent. The active fraction of identified active asteroids greater than 1 km to main belt asteroids greater than 1 km is $f \, \sim \, 10^5$, and describes a lower limit as many objects are yet undetected. \cite{in AA, may be other source} From a study of 30,000 objects observed near perihelion, \cite{} Hsieh et all (2015) concluded that $f \, \sim \, 10^4$ for asteroids which are active at any instant in the outer belt.


Observations preformed by the Outer Solar System Origins Survey (OSSOS) with the Canada-France-Hawaii telescope (CFHT) MegaPrime camera have been colected since 2013. The MegaPrime camera consists of a 36 CCD image plane, each 2048x4125 CCD with sensitivity of 0.185"/pix. This covers a field of roughly 1$^{\circ}$ x 1$^{\circ}$ on the sky. For this study, images taken with r (and u) filters, and exposure times greater than 287s were used to make detections.






\acknowledgments{
  Acknowledgments. 
}

\end{document}

\bibliographystyle{plainnat}
\bibliography{mbc_paper}