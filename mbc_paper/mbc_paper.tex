\documentclass[iop,apj]{emulateapj}
\usepackage{amsmath,amssymb,amstext}

\usepackage[breaklinks,colorlinks,citecolor=blue,linkcolor=magenta]{hyperref} 
\renewcommand*{\sectionautorefname}{Section}
\usepackage[all]{hypcap} %Links go to figures; breaks on deluxetables (use \capstartfalse \capstarttrue to fix it)

\usepackage{aas_macros}
\usepackage{natbib}
\bibliographystyle{apj}

\shorttitle{Short Title}
\shortauthors{Author et al.}

\begin{document}

\title{Detecting Active Asteroids/Comets from OSSOS survey images}
\author{Authors}
\affil{Affiliations}

\begin{abstract}
Abstract.
\end{abstract}

\keywords{keywords}
\maketitle

\section{Introduction}
% so far, im not making a point of using good wording
% definietly have to go back and add in citations

The active asteroids are small bodies in the main asteroid belt which have transient dust emission producing comet-like comae and tails. Unlike comets, which originate in the Kuiper Belt and Oort cloud and have been scattered inwards by gravitational affects with the large planets, the active asteroids have been stable orbits confined to the main belt and likely formed in the same location as they reside presently \cite{TEST}(pop of comets 2006). For objects which formed in the the outer region of the main belt, beyond the snow line, the crystallized water ice which was present at the time of formation and not exposed to primordial heating may still remain in reservoirs beneath the surface \cite{TEST}(limits on size, prialnik 2009).  According to models done by \citet{fanale89} Fanale and Salvail (1989),  beyond heliocentric distances of 2.4 AU ice can be protected against sublimation by a "relatively thin" surface regolith  of depth 1 -- 100 m for the entire age of the solar system. If the ice layer were to be exposed to sub solar heating,  sublimation could be triggered,  ejecting dust particles from the surface producing a comae.  %Objects in the main asteroid belt which are driven by sublimation are also named Main Belt Comets (MBCs). 
Dust emission could also be a result of an impact, rotational instabilities due to YORP torques, or a combination of several effects \cite{hiesh2015}%; these objects are known as disrupted asteroids. 
The ejected dust would form a coma around the object, and radiation pressure affects would then (preferentially) push the smaller particles away from the asteroid forming a tail.

%Sublimation driven active asteroids differ from comets in that the comets, having larger ice reservoirs, would have stronger sublimation events which could eject larger debris. The cometary tail would then be longer lived as the larger debris would be less effected by the radiation pressure, and thus dissipate slowly. It is also possible, however, that prolonged activity is a result of ongoing ejection of small, fast dissipating particles. \cite{TEST} 

Since the first discovery of an active main-belt asteroid, 133P/Elst-Pizarro, several attempts have been made to identify new objects of this type. A comprehensive review of such searches can be found in \citet{hsieh2015}(panstarrs perspective).  A persistent challenge to this effort is that the detection of the coma or tails is highly dependent on the magnitude constraints of the survey for small dark objects. As most asteroids fall near the limiting magnitudes of the survey in which they are discovered \cite{jewitt15}, objects which are larger, closer, or have higher albedo are preferentially detected and any dust emission would be more easily apparent. The active fraction of identified active asteroids greater than 1 km to main belt asteroids greater than 1 km is $f \, \sim \, 10^5$, and describes a lower limit as many objects are yet undetected. \cite{jewitt15}(AA, check if other source) %From a study of 30,000 objects observed near perihelion, \cite{} Hsieh et all (2015) concluded that $f \, \sim \, 10^4$ for asteroids which are active at any instant in the outer belt.


% WHAT IS NEW, BETTER, DIFFERENT ABOUT OSSOS???

Observations taken by the Outer Solar System Origins Survey (OSSOS) with the Canada-France-Hawaii telescope (CFHT) MegaPrime wide-field optical imaging facility have been collected since 2013. The wide-field images, MegaCam, consists of a 36 CCD image plane, each 2048 x 4125 CCD with resolution of 0.185"/pix. This covers a field of  1$^{\circ}$ x 1$^{\circ}$ on the sky at the summit of Mauna Kea, Hawaii. For this study, images taken with r (and u) filters, and exposure times greater than 287s were used to make detections.






\acknowledgments{
  Acknowledgments. 
}

\bibliographystyle{plainnat}
\bibliography{mbc_paper}

\end{document}



